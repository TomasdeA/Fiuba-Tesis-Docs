%------------------------- Carga de paquetes ---------------------------
%
% Si no necesitas algún paquete, comentalo.
%
\usepackage{indentfirst}

%
% Definición del tamaño de página y los márgenes:
%
%\usepackage[a4paper,headheight=16pt,scale={0.7,0.8},hoffset=0.5cm]{geometry}

%
% me permite definir los margenes como quiera
%
\usepackage{anysize}
% MARGENES
\marginsize{2cm}{2cm}{1cm}{1.5cm} %izquierda, derecha, arriba, abajo

%
% Vamos a escribir en castellano:
%
\usepackage[spanish,es-tabla]{babel}

% NO FUNCIONA CON LA VERSION TeX Live 2022 cuando a la vez se compila hyperref. Asi que lo comento
% Fuente: Arial 12 xD ( https://www.overleaf.com/blog/tex-live-2022-now-available )
%\usepackage[utf8x]{inputenc}

%
% Si preferís el tipo de letra Helvetica (Arial), descomentá las siguientes
% dos lineas (las fórmulas seguirán estando en Times):
%
%\usepackage{helvet}
%\renewcommand\familydefault{\sfdefault}

%
% El paquete amsmath agrega algunas funcionalidades extra a las fórmulas. 
% Además defino la numeración de las tablas y figuras al estilo "Figura 2.3", 
% en lugar de "Figura 7". (Por lo tanto, aunque no uses fórmulas, si querés
% este tipo de numeración dejá el paquete amsmath descomentado).
% Todos los paquetes que estan aca agregan funcionalidad a math.

\usepackage{amsmath, amsthm, amssymb}
\usepackage{mathtools}
\usepackage{cancel} % Para cancelar en formulas matemáticas
\usepackage{bbm} %Agrega \mathbbm{} que permite aplicar \mathbb{} números (Función indicadora)
\usepackage{mathrsfs} % Agrega \mathscr{} que ofrece mas fonts
\usepackage{mathdots}
\usepackage[version=4]{mhchem}
\usepackage{esint}
\usepackage{stackrel}
\usepackage{stmaryrd}
\usepackage{cases}

% Descomenta si lo necesitas
%\numberwithin{equation}{section}


%
% Para tener cabecera y pie de página con un estilo personalizado:
%
\usepackage{fancyhdr}

%
% Para poner el texto "Figura X" en negrita:
% (Si no tenés el paquete 'caption2', probá con 'caption').
%
\usepackage[bf]{caption}

%
% Para poder usar subfiguras: (al estilo Figura 2.3(b) )
%
%\usepackage{subfig}
\usepackage{subcaption} % Otra Opción, dicen que es mejor que subfig

%
% Para poder agregar notas al pie en tablas:
%
\usepackage{threeparttable}

%
% Para poder poner las imagen es donde yo quiero.
%
\usepackage{float}


%
% graphicx
%
%%------------------------------ graphicx ----------------------------------
%
% Para incluir imágenes, el siguiente código carga el paquete graphicx 
% según se esté generando un archivo dvi o un pdf (con pdflatex). 
%
\newif\ifpdf
\ifx\pdfoutput\undefined
	\pdffalse
\else
	\pdfoutput=1
	\pdftrue
\fi

\ifpdf
	\usepackage[pdftex]{graphicx}
	\pdfcompresslevel=9
\else
	\usepackage[dvips]{graphicx}
\fi

%
% Todas las imágenes están en el directorio Imagenes:
%
\newcommand{\imgdir}{_Imagenes}
\graphicspath{{\imgdir/}}
%
%------------------------------ graphicx ---------------------------------- %En teoria esto ya no hace mas falta (MODO BETA)
\usepackage{graphicx}

%
% Para poner imagenes .svg [ \includesvg[options]{imagefilename} ] (Se puede usar en figure)
%
\usepackage{svg}

%
% Todas las imágenes están en el directorio _Imagenes:
%
\newcommand{\imgdir}{_Imagenes}
\graphicspath{{\imgdir/}}
%


%
% extiende el paquete de graphicx
%
\usepackage{adjustbox}


%
% para justificar el texto
%
\usepackage{ragged2e}



%
% letras griegas sin cursiva
%
\usepackage{upgreek}


%
% SI
%
\usepackage{siunitx}
\sisetup{
output-decimal-marker = {,},
per-mode = fraction, % El que esta por default en el paquete es reciprocal, ej: m.s^-1
exponent-product = \cdot ,
output-complex-root=\ensuremath{\mathrm{j}},
complex-root-position=before-number,
}


%
% el paquete pone el footnote en el fondo de la pagina
%
\usepackage[bottom]{footmisc} 
\setlength{\headheight}{28pt} %config del paquete

%
% Permite poner multicolumnas mediantes:  \begin{multicols}{numero_columnas} ...   \end{multicols}
%
\usepackage{multicol,multirow}

%
% parecido a listing pero no pensado para codigo, puede crear el env comment para comentar
%
\usepackage{verbatim}

%
% Paquete para manejo de csv
%
\usepackage{csvsimple} 


%
% Paquete para hacer lineas diagonales en tablas
%
\usepackage{diagbox}

%
% Paquete para hacer quotes
%
\usepackage{dirtytalk}

%
% imágenes embebidas
%
\usepackage{wrapfig}


%
% Paquete para plotear funciones en latex
%
\usepackage{pgfplots}
\pgfplotsset{compat=newest}

%
% Paquete para crear figuras gráficas (Poderoso, pero medio complicado de usar)
%
\usepackage{tikz,pgf}
\usetikzlibrary{positioning}
\usetikzlibrary{circuits.plc.ladder} % L A TEX and plain TEX when using Tik Z


%
%  Create common layout for tabular material.
%
\usepackage{makecell}

%
% Enumerar alfabeticamente (a), (b),... etc
%
\usepackage{enumitem}

%
% https://www.ctan.org/pkg/tabularx
%
\usepackage{tabularx}

%----------------------------------------------------------------------------------------

%
% Para escribir ejercicios en los documentes. Define \begin{exercise}...\end{exercise}
%
\usepackage{exercise}

%
%
%
\usepackage{xcolor} % Para los colores
% Configuracion de xcolor, defino un par de colores
\definecolor{codegreen}{rgb}{0,0.6,0}
\definecolor{codegray}{rgb}{0.5,0.5,0.5}
\definecolor{codepurple}{rgb}{0.58,0,0.82}
\definecolor{backcolour}{rgb}{0.95,0.95,0.92}
\definecolor{MyDarkGreen}{rgb}{0.0,0.4,0.0} % This is the color used for comments in AVR.tex


%--------------------------------------------------------------------------------------
\usepackage{listings} % Para poner codigo en latex, queda lindo
% \input{__Header/Listings/AVR}
% \input{__Header/Listings/PowerShell}
%Configuracion de listings
\lstdefinestyle{mystyle}{
    backgroundcolor=\color{backcolour},   
    commentstyle=\color{codegreen},
    keywordstyle=\color{magenta},
    numberstyle=\tiny\color{codegray},
    stringstyle=\color{codepurple},
    basicstyle=\ttfamily\footnotesize,
    breakatwhitespace=false,         
    breaklines=true,                 
    captionpos=b,                    
    keepspaces=true,                 
    numbers=left,                    
    numbersep=5pt,                  
    showspaces=false,                
    showstringspaces=false,
    showtabs=false,                  
    tabsize=2
}

% Descomenta si queres usar este estilo fijo en el documento
%\lstset{style=mystyle}
\usepackage{matlab-prettifier} % Uso → \lstinputlisting[style=Matlab-editor]{sample.m}

%--------------------------------------------------------------------------------------
% Otro paquete para poner código en látex. Soporta VHDL 
%\usepackage{minted}

%--------------------------------------------------------------------------------------
% entornos gráficos encapsulados (para enunciados) 
\usepackage{framed}
\usepackage{mdframed} % extensión del paquete framed.
\mdfdefinestyle{Ireland}{%
    linecolor=green!20!white,
    outerlinewidth=2pt,
    roundcorner=200pt,
    innertopmargin=\baselineskip,
    innerbottommargin=\baselineskip,
    innerrightmargin=20pt,
    innerleftmargin=20pt,
    backgroundcolor=green!20!white
}
%--------------------------------------------------------------------------------------
\usepackage{nicematrix}

%--------------------------------------------------------------------------------------
%Para la bibliografia
\usepackage[backend=biber,style=ieee]{biblatex}
\usepackage{csquotes}
\addbibresource{bibliografia.bib}
%---------------------------------------------------------------------------------------
\usepackage{hyperref} %Paquete para hyperlinks
% Poner siempre este paquete al final.
%Configuración del paquete hyperref
\hypersetup{
    colorlinks=true,
    linkcolor=blue,
    filecolor=magenta,      
    urlcolor=cyan}

%---------------------------------------------------------------------------------------
% Poner este paquete despues de Hyperref SIEMPRE.
%
% Mejora y hace mas cómodo hacer referencias con \cref{}
%
\usepackage{cleveref}



%---------------------------------------------------------------------------------------
%
% Te deja rotar tablas
%
\usepackage{rotating}


%
% Cosas lidas de tablas
%
\usepackage{booktabs}
\usepackage{array}
\usepackage{ragged2e}   % <-- para \RaggedRight
\newcolumntype{L}[1]{>{\RaggedRight\arraybackslash}p{#1}}