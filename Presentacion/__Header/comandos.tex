%:::::::::::::::::::::::::::::::::::::::::::::::::::::::::::::::::::::::::::::::::::::::::
% Cosas misceláneas que no sabia donde meter
%:::::::::::::::::::::::::::::::::::::::::::::::::::::::::::::::::::::::::::::::::::::::::

% Evitar warnings de underfull
\hbadness=99999

% Define el comando \nota{#} el cual pone "#" en texto rojo automaticamente. Me sirve para dejar notas por el trabajo para que no pasen desapercibidas.
\newcommand{\nota}[1]{ \textcolor{red}{#1} }

\newcommand{\ladderlabel}[1]{ \scriptsize{\texttt{\%#1}} }

%:::::::::::::::::::::::::::::::::::::::::::::::::::::::::::::::::::::::::::::::::::::::::
% Comandos útiles
%:::::::::::::::::::::::::::::::::::::::::::::::::::::::::::::::::::::::::::::::::::::::::


% Paréntesis: \pa{algo}, tamaño variable: \pa*{}
\DeclarePairedDelimiter{\pa}{\lparen}{\rparen}
% Corchetes: \br{algo}, tamaño variable: \br*{}
\DeclarePairedDelimiter{\br}{[}{]}
% Llaves: \cbr{algo}, tamaño variable: \cbr*{}
\DeclarePairedDelimiter{\cbr}{\{}{\}}
%
\DeclarePairedDelimiter{\expval}{\langle}{\rangle}
% Otros
\DeclarePairedDelimiter{\eval}{ . }{\rvert}
\DeclarePairedDelimiter{\floor}{\lfloor}{\rfloor}
\DeclarePairedDelimiter{\ceil}{\lceil}{\rceil}

%% Operadores
\DeclarePairedDelimiter{\abs}{\lvert}{\rvert} % Modulo, tamaño variable: \abs*{}
\DeclarePairedDelimiter{\dabs}{\|}{\|} % Norma, tamaño variable: \dabs*{}
\newcommand{\E}[1]{\ensuremath{\operatorname{\mathbb{E}}\left[ #1 \right]}} % Esperanza
\newcommand{\var}[1]{\ensuremath{\operatorname{Var}\left( #1\right)}} % Varianza
\newcommand{\cov}[1]{\ensuremath{\operatorname{C}\left( #1\right)}} % Covarianza
\newcommand{\prob}[1]{\ensuremath{\operatorname{\mathbb{P}}\left( #1 \right)}} %Probabilidad

%% Matrices
\newcommand{\m}[1]{\ensuremath{\begin{matrix} #1 \end{matrix}}}
\newcommand{\sm}[1]{\ensuremath{\begin{smallmatrix} #1 \end{smallmatrix}}}
\newcommand{\bm}[1]{\ensuremath{\begin{bmatrix} #1 \end{bmatrix}}} % Para crear matrices con []
\newcommand{\bsm}[1]{\ensuremath{\begin{bsmallmatrix} #1 \end{bsmallmatrix}}} % Para crear matrices con [] chicas
\newcommand{\ppm}[1]{\ensuremath{\begin{pmatrix} #1 \end{pmatrix}}} % Para crear matrices con ()
\newcommand{\psm}[1]{\ensuremath{\begin{psmallmatrix} #1 \end{psmallmatrix}}} % Para crear matrices con () chicas

%% Sumatoria (\sumatoria{desde}{hasta}{expresión})
\newcommand{\sumatoria}[3]{\ensuremath{\sum_{#1}^{#2} #3}}

%% Productoria (\productoria{desde}{hasta}{expresión})
\newcommand{\productoria}[3]{\ensuremath{\prod_{#1}^{#2} #3}}

%% Otros
\newcommand{\ds}{\displaystyle}
\newcommand{\on}{\operatorname}
\newcommand{\p}{\partial}
\newcommand{\dd}{\mathop{}\!\mathrm{d}}
\newcommand{\ts}{\textsuperscript}
\newcommand{\tsub}{\textsubscript}

% Derivadas totales
\newcommand{\dv}[3][]{\frac{\dd^{#1} #2}{\dd #3^{#1}}} % Ej: $\dv[orden]{f}{respecto de}$
% Derivadas parciales
\newcommand{\pdv}[3][]{\frac{\p^{#1} #2}{\p #3^{#1}}} % Ej: $\pdv[orden]{f}{respecto de}$

% Agregar numero de ecuacion inline
\newcommand\inlineeqno{\stepcounter{equation}\ (\theequation)}
% uso: $ e_{33} = 0 \inlineeqno $

% Macro para el simbolo del electron
\newcommand{\el}{e^{\text{-}}}


